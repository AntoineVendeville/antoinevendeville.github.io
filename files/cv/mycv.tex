\documentclass[a4paper,12pt]{article}

\usepackage[utf8x]{inputenc}
\usepackage[scaled]{helvet} % Load Helvetica font
\renewcommand{\familydefault}{\sfdefault} % Set sans-serif as default
\usepackage[T1]{fontenc}

\usepackage{xcolor}

%\usepackage[french]{babel}

\usepackage{url,xcolor,titlesec,graphicx,longtable,etaremune}
\usepackage[a4paper]{geometry}
\geometry{left=2.8cm,right=2.8cm,top=2.5cm,bottom=2.5cm}
\usepackage{fouriernc}
\usepackage[pagebackref = false]{hyperref}
\hypersetup{
  colorlinks = true,
  urlcolor = blue, 
  linkcolor = blue,
  citecolor = blue,
  pdftitle = {CV Antoine Vendeville},
  pdfauthor = {Antoine Vendeville},
  pdfsubject = {CV Antoine Vendeville}
}


\titleformat*{\section}{             
  \center
  \large \scshape%\bfseries
}

\titleformat*{\subsection}{
  %\center
  \bfseries %\scshape 
}

\titleformat*{\subsubsection}{
  \center \itshape 
}

\setlength{\parindent}{0cm}

\begin{document}
%\hrule
\bigskip
\begin{center}
\LARGE\textsc{Curriculum Vit\ae}

\medskip

%\small{(Mise à jour : \today)}
\end{center}
\bigskip

%\hrule
\bigskip
\bigskip

\thispagestyle{empty} % no page numbering
 
{\large\textbf{Antoine VENDEVILLE}}
\vspace{1cm}

Born on April 23\textsuperscript{rd}, 1994 in Paris, France.

\medskip

\textbf{Post-doctoral researcher}\\ 
\href{https://medialab.sciencespo.fr/en/}{Médialab, 
Sciences Po Paris}

\medskip
\textbf{Supervisory team}\\
\href{https://pedroramaciotti.github.io/}{Pedro Ramaciotti Morales}

\medskip

\textbf{Other affiliations}\\
\href{https://iscpif.fr/?lang=en}{Institute of Complex Systems, Paris}\\
\href{https://www.learningplanetinstitute.org/en/}{Learning Planet Institute, Paris}\\
\medskip

\textbf{Contact}\\
  \href{mailto:antoine.vendeville@sciencespo.fr}{antoine.vendeville@sciencespo.fr}

  \href{https://antoinevendeville.github.io}{Personal webpage}

  \href{https://scholar.google.com/citations?user=D7pFu_8AAAAJ&hl=fr}{Google Scholar profile}


\renewcommand{\baselinestretch}{1.0}

\section*{Academic Positions}
\subsection*{2023-2025} 
Post-doctoral researcher at \href{https://medialab.sciencespo.fr/}{Médialab Siences Po}, Paris. \href{https://medialab.sciencespo.fr/activites/ai-political-machine/}{AI-Political Machines} project, supervised by \href{https://pedroramaciotti.github.io/}{Pedro Ramaciotti Morales}. Research topics: mathematical modeling of political opinions, algorithmic recommendations, regulation of online social platforms. I am also affiliated with the \href{https://iscpif.fr/?lang=en}{Institute of Complex Systems of Paris} and the \href{https://www.learningplanetinstitute.org/en/}{Learning Planet Institute}. I am also a research associate at the \href{https://www.digitalspeechlab.com/}{Digital Speech Lab} in University College London.

\subsection*{2019-2023} 
PhD student at \href{https://www.ucl.ac.uk/}{University College London}, Department of Computer Science, Centre for Doctoral Training in Cybersecurity. The echo chamber effect in social networks: theoretical analysis and steering strategies. Supervised by Dr.\ \href{https://bguedj.github.io/}{Benjamin Guedj} and Dr.\ \href{https://wp.cs.ucl.ac.uk/shizhou/}{Shi Zhou}. %Thesis title: \textit{The echo chamber effect in social networks: theoretical analysis and steering strategies.} Available \href{https://discovery.ucl.ac.uk/id/eprint/10185036/}{online}.


\subsection*{2019} 
Research assistant, Sorbonne Université. Department of Computer Science (\href{https://www.lip6.fr/}{LiP6}), team \href{https://www-npa.lip6.fr/}{NPA}. Supervised by \href{https://anastasiosgiovanidis.net/}{Anastasios Giovanidis}, \href{https://www.lip6.fr/actualite/personnes-fiche.php?ident=P144}{Bruno Baynat}, \href{https://www-complexnetworks.lip6.fr/~magnien/}{Clémence Magnien}. Duration 5 months.

\subsection*{2018} 
Research assistant, Sorbonne Université. Department of Computer Science (\href{https://www.lip6.fr/}{LiP6}), team \href{https://www-npa.lip6.fr/}{NPA}. Supervised by \href{https://anastasiosgiovanidis.net/}{Anastasios Giovanidis}, \href{https://www.lip6.fr/actualite/personnes-fiche.php?ident=P144}{Bruno Baynat}, \href{https://www-complexnetworks.lip6.fr/~magnien/}{Clémence Magnien}. Duration 5 months.


% \section*{Other affiliations}
% $\bullet$ \href{https://iscpif.fr/?lang=en}{Paris Institute of Complex Systems}.

% $\bullet$ \href{https://www.digitalspeechlab.com/}{Digital Speech Lab} at University College London. %directed by \href{https://jeffreywhoward.com/}{Jeffrey Howard}


\section*{Research visits}
\textbf{2023} \href{https://ifisc.uib-csic.es/en/}{IFISC}, University of the Balearic Islands, Palma de Mallorca (1 week).

%\textbf{2023} \href{https://sailab.diism.unisi.it/}{SAILab}, University of Siena, Italy (2 months).

\textbf{2022} \href{https://www.lip6.fr/}{LiP6}, Sorbonne Université, Paris (3 months).


\section*{Publications}

\subsection*{2025}
$\bullet$ A. Vendeville. A model for French voters. Accepted for publication in Physical Review E. \href{https://arxiv.org/abs/2501.13215}{arXiv:2501.13215}.

$\bullet$ A. Vendeville, F. Diaz-Diaz. Echo chamber effects in signed networks. \href{https://doi.org/10.1103/PhysRevE.111.024302}{Physical Review E, 111(2), 024302.}

\subsection*{2024}
$\bullet$ A. Vendeville. The echo chamber effect in social networks: theoretical analysis and steering strategies. \href{https://discovery.ucl.ac.uk/id/eprint/10185036/}{PhD thesis.}

\subsection*{2023}
$\bullet$ A. Vendeville, S. Zhou and B. Guedj. Discord in the voter model for complex networks. \href{https://journals.aps.org/pre/abstract/10.1103/PhysRevE.109.024312}{Phyical Review E, 109(2), 024312.}

$\bullet$ A. Vendeville, A. Giovanidis, E. Papanastasiou and B. Guedj. Opening up echo chambers via optimal content recommendations. \href{https://doi.org/10.1007/978-3-031-21127-0_7}{Proceedings of Complex Networks and Their Applications XI, 2023.}

\subsection*{2022}
%$\bullet$ A. Vendeville, A. Giovanidis, E. Papanastasiou and B. Guedj. Recommendation of content to mitigate the echo chamber effect. Extended abstract. Accepted at \href{https://www.ccs2022.org/}{CCS 2022}.

%$\bullet$ A. Vendeville, B. Guedj and S. Zhou. Active links density in the Voter Model with zealots. Extended abstract. Accepted at \href{https://www.ccs2022.org/}{CCS 2022}.

$\bullet$ A. Vendeville, B. Guedj and S. Zhou. Towards control of opinion diversity by introducing zealots into a polarised social group. \href{https://doi.org/10.1007/978-3-030-93413-2_29}{Proceedings of Complex Networks and Their Applications X, 2022.} %Preprint on \href{https://arxiv.org/abs/2006.07265}{arXiv}, \href{https://hal.inria.fr/hal-02872161}{HAL}.

\subsection*{2021}
$\bullet$ A. Giovanidis, B. Baynat, C. Magnien and A. Vendeville. Ranking Online Social Users by Their Influence. \href{https://doi.org/10.1109/TNET.2021.3085201}{IEEE/ACM Transactions on Networking, 2021.} %Preprint on \href{https://arxiv.org/abs/2107.01914}{arXiv}, \href{https://hal.archives-ouvertes.fr/hal-02970215}{HAL}.

$\bullet$ A. Vendeville, B. Guedj and S. Zhou. Forecasting elections results via the voter model with stubborn nodes. \href{https://doi.org/10.1007/s41109-020-00342-7}{Applied Network Science 6, 1, 2021.} %Preprint on \href{https://arxiv.org/abs/2009.10627}{arXiv}, \href{https://hal.archives-ouvertes.fr/hal-02946434}{HAL}.

\subsection*{2019}
$\bullet$ A. Giovanidis, B. Baynat and A. Vendeville. Performance Analysis of Online Social Platforms. \href{https://ieeexplore.ieee.org/abstract/document/8737539}{Proceedings of IEEE Conference on Computer Communications, 2019.} %Preprint on \href{https://arxiv.org/abs/1902.07187}{arXiv}, \href{https://hal.archives-ouvertes.fr/hal-01941296}{HAL}.


\section*{Working papers}
$\bullet$ A. Vendeville, J. Royo-Letelier, D. Cassells, J.-P. Cointet, T. Faverjon, T. Lenoir, B. Mazoyer, B. Ooghe-Tabanou, A. Pournaki, H. Yamashita et P. Ramaciotti. Population of X/Twitter users and web domains embedded in a multidimensional political opinion space. Submitted to Nature Scientific Data.

% $\bullet$ D. Cassells, A. Vendeville, L. Tabourier, P. Ramaciotti. The Role of Group Identity in Opinion Polarization. Submitted to Journal of Artificial Societies and Social Simulation.

$\bullet$ G. Di Bona, E. Fraxanet, B. Komander, A. Lo Sasso, V. Morini, A. Vendeville, M. Falkenberg, A. Galeazzi. Sampled Datasets Risk Substantial Bias in the Identification of Political Polarization on Social Media. \href{https://arxiv.org/abs/2406.19867}{arXiv:2406.19867}.

% \section*{Talks and Presentations}

% \subsection*{2023}
% $\bullet$ Echo chambers and opinion diversity in the Voter Model: towards regulation strategies for social networks. Presented on November 7, 2023 at the \href{https://ifisc.uib-csic.es/en/}{IFISC} in Palma de Mallorca.

% $\bullet$ Opening up Echo Chambers via Optimal Content Recommendation. Presented on April 25, 2023 at the \href{https://medialab.sciencespo.fr/}{MédiaLab Sciences Po Paris}.

% $\bullet$ Opening up Echo Chambers via Optimal Content Recommendation. Presented on April 20, 2023 at the \href{https://sites.google.com/view/netplace}{NetPLACE}.

% $\bullet$ Opening up Echo Chambers via Optimal Content Recommendation. Presented on April 14, 2023 at the Data Natives seminar at City University, London.


% \subsection*{2022}

% $\bullet$ Opening up Echo Chambers via Optimal Content Recommendation. Presented on November 8, 2022 at \href{https://complexnetworks.org/}{Complex Networks 2022}.

% $\bullet$ Active links in the voter model with zealots, poster presented on October 20, 2022 at \href{https://ccs2022.org/}{Conference on Complex Systems}.

% $\bullet$ Recommendation of content to mitigate the echo chamber effect, poster presented on October 17, 2022 at \href{https://ccs2022.org/}{Conference on Complex Systems}.

% $\bullet$ Opening up Echo Chambers via Optimal Content Recommendation. Presented on 29, 2022 at Centre d’Economie de la Sorbonne, for the \href{https://sites.google.com/site/cesworkinggroupnetworks/}{Networks and Games seminar}.

% $\bullet$ Fighting political echo chambers via content recommendation: Method and application to the 2017 French presidential elections, presented on June 29, 2022 at the Mediterranean School of Complex Networks \href{https://mediterraneanschoolcomplex.net/}{(MSCX)}.

% $\bullet$ Depolarising Social Networks: Optimisation of Exposure to Adverse Opinions in the Presence of a Backfire Effect, presented on April 12, 2022 at Sorbonne Université to the \href{https://www.complexnetworks.fr/}{Complex Networks} and \href{https://www-npa.lip6.fr/}{NPA} teams.

% $\bullet$ Voter model with zealots for opinion control and forecast of election results, presented on January 7, 2022 at University College London, for the Information and Decision Systems research group (\href{https://www.ucl.ac.uk/computer-science/research/research-groups/information-and-decision-systems-ids}{IDS}).

% \subsection*{2021}
% $\bullet$ Towards control of opinion diversity by introducing zealots into a polarised social group, poster presented on December 1, 2021 at \href{https://complexnetworks.org/}{Complex Networks 2021}.

% $\bullet$ Voter Model with Stubborn Agents: from Theoretical Solutions to Prediction of Political Elections, presented on July 5, 2021 at \href{https://networks2021.net/}{Networks 2021}.

\section*{Supervision}
\textbf{2021} \href{https://bguedj.github.io/}{Benjamin Guedj} and I supervised \href{https://valentinkil.github.io/}{Valentin Kilian} for a 2 months research internship on opinion dynamics at \href{https://www.inria.fr/en}{Inria}.


\section*{Teaching}
\textbf{2019-2023} Teaching assistant at University College London for \href{https://www.ucl.ac.uk/module-catalogue/modules/complex-networks-and-web/COMP0123}{Complex Networks and Web (COMP0123)}. 30h per academic year, 120h total.

\textbf{2020-2021} Teaching assistant at University College London for \href{https://www.ucl.ac.uk/module-catalogue/modules/introductory-programming/COMP0066}{Introductory Programming (COMP0066)}. 30h.

\textbf{2013-2017} Private tutor of mathematics for high school students in Paris.

\section*{Academic engagement}
I was a co-organiser of the 2025 edition of the \href{https://wwcs2025.github.io/}{Winter Workshop on Complex Systems}. I was a student volunteer at \href{https://infocom2019.ieee-infocom.org/index.html}{INFOCOM 2019}. I have reviewed papers for the following journals and conferences.
\begin{itemize}
  \item[$\bullet$] \href{https://www.ic2s2-2025.org/}{IC2S2}.% (6 abstracts, 2025).
	\item[$\bullet$] \href{https://www.nature.com/srep/}{Nature Scientific Reports}.% (5 papers, 2020-2022).
	\item[$\bullet$] \href{https://journals.plos.org/plosone/}{PLOS ONE}.% (1 paper, 2022).
	\item[$\bullet$] \href{https://www.journals.elsevier.com/computer-communications}{Computer Communications}.% (1 paper, 2022).
	\item[$\bullet$] \href{https://royalsocietypublishing.org/journal/rspa}{Proceedings of the Royal Society A}.% (1 paper, 2021).
\end{itemize}

\section*{Funding}
My post-doctoral research project at Sciences Po is funded by the \href{https://www.sciencespo.fr/nous-soutenir/fr/nos-projets/developper-la-recherche/project-liberty-s-institute-mc-court-institute/}{McCourt institute}. My PhD was funded by the \href{https://www.ukri.org/councils/epsrc/}{UK EPSRC}. In 2023, I received the \href{http://yrcss.cssociety.org/bridge-grants/}{yrCSS Bridge Grant}, which fosters collaborations between young researchers.

\section*{Education}
\textbf{2018-2019} Master 2 Data Science, Université Claude Bernard, Lyon.

\textbf{2017-2018} Master 2 Mathematical Modelling, Sorbonne Université.

\textbf{2015-2017} Master 1 Mathematics and Applications, Sorbonne Université.

\textbf{2012-2015} Bachelor in Mathematics and Applications, Sorbonne Université.%\footnote{\label{note1}Former Université Pierre et Marie Curie, Paris, France.}.


\thispagestyle{empty} % no page numbering

\end{document}
